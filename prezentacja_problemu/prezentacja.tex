\documentclass[10pt]{beamer}
\usetheme{Warsaw}

\usepackage[polish]{babel}
\usepackage[utf8]{inputenc}
\usepackage[T1]{fontenc}

\usepackage{parskip}
\usepackage{latexsym,gensymb,amsmath,amssymb,amsthm}
\usepackage{graphicx}
\usepackage{url}

\usepackage{graphics}
\usepackage{graphicx}
\usepackage{hyperref}

\title[Flow Shop Scheduling Problem]{Flow~Shop~Scheduling~Problem}
\author[Krzysztof Chrobak\and Jan Sochiera]{Krzysztof~Chrobak \and Jan~Sochiera}
	

\institute[Algorytmy Ewolucyjne 2012]{\normalsize Algorytmy Ewolucyjne 2012}
\subject{Computational Sciences}


\begin{document}

  \frame
  {
    \titlepage
  }



  \section{Definicja problemu}


  \frame
  {
    \frametitle{Definicja problemu}

    \begin{itemize}
    	\item<1-> $n$ zadań
    	\item<2-> $m$ maszyn
	\item<3-> Każde zadanie musi być wykonane na każdej maszynie
	\item<4-> Każda maszyna musi wykonaywać zadania w tej samej kolejności
	\item<5-> Jedno zadanie nie może być wykonywane jednocześnie na więcej niż jednej maszynie
    \end{itemize}
  }

  \section{Przestrzeń poszukiwań}

\frame
	{
	\frametitle{Przestrzeń poszukiwań}
	\begin{center}
	Każda permutacja kolejności zadań jest możliwa do wykonania, dlatego przestrzeń poszukiwań staje się przestrzenią permutacji zbioru $n$-elementowego.  
	\end{center}
	}

\section{Funkcja celu}
\frame
	{
	\frametitle{Funkcja celu}
	\begin{description}
		
		\item[$t_{i,j}$] Czas wykonywania $i$-tego zadania na $j$-tej maszynie
		\item[$S_n$] Przestrzeń poszukiwań
  		\item[$\pi \in S_n$] Kolejność wykonywnia zadań
  		\item[$b(i,j)$] Czas rozpoczęcia $\pi(i)$-tego zadania na $j$-tej maszynie
  		\item[$c(\pi)$] Czas pomiędzy początkiem pierwszego a końcem ostatniego zadania \hfill \\
				$c(\pi) = b(n,m) + t_{\pi (n), m}$
		
	\end{description}
	}

\frame
	{
	\frametitle{Algorytm obliczania funkcji celu}
	
	$b(1,1) = 0$\\~\\
	$b(1,n) = b(1,n-1) + t_{\pi(1),n-1}$\\~\\
	$b(i,1) = b(i-1,1) + t_{\pi(i-1),1}$\\~\\
	$b(i,j) = max(b(i-1, j) + t_{\pi(i-1),j},$\\$\;\;\;\;\;\;\;\;\;\;\;\;\;\;\;\;\;\;\;\;\;b(i, j-1) + t_{\pi(i),j-1})$
	
	}

\end{document}








